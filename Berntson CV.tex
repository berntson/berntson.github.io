\documentclass{cv_07152020}
\usepackage{tabularx}
\usepackage{etaremune}
\begin{document}

%Figure out how to wrap descriptions.
%Add links.
%Is there some way to automatically add publications	
	
\noindent{\Huge\textbf{Daniel Berntson}}

\vspace{20pt}
\noindent Department of Philosophy	\hfill \href{http://www.danielberntson.com}{www.danielberntson.com}\newline
\noindent Princeton University	\hfill \href{berntson@princeton.edu}{berntson@princeton.edu} \newline
\noindent Princeton, NJ 08544 \hfill  \newline
\\
\noindent Home: (401) 484-7893 \par
\noindent Placement: (609) 258-6161 \par


\section*{Employment}
\medskip
\begin{etaremune}
\item[] Rutgers University, postdoc, 2020-present
\item[] Princeton University, lecturer, 2019-2020
\end{etaremune}


\section*{Education}
\medskip
\begin{etaremune}
	\item[] Princeton University, PhD in Philosophy, 2010-19
	\item[] Brown University, MA in Philosophy, 2007-10
	\item[] Northwestern College, BA in Philosophy and English Literature, 2002-06
\end{etaremune}


\section*{Research}
\medskip
\begin{etaremune}
	\item[] AOS: Metaphysics, Philosophy of Science
	\item[] AOC: Epistemology, Logic, Buddhist Philosophy
\end{etaremune}

\section*{Honors}
\medskip
\begin{etaremune}
	\item[] \href{https://gradschool.princeton.edu/costs-funding/sources-funding/fellowships/competitive}{Harold W. Dodds Honorific Fellowship}, Princeton University, 2014-15
	\item[] Princeton University Fellowship, 2010-14
	\item[] American Graduate Fellowship, \href{https://www.cic.edu}{Council of Independent Colleges}, 2008-10	
	\item[] Norman Vincent Peale Scholarship, Northwestern College, 2002-06
\end{etaremune}


\section*{Publications}
\medskip
\begin{etaremune}
\item ``A New Prospect for Epistemic Aggregation'' (with Yoaav Isaacs), Epistemé 10 \mbox{(3) :} 269-28 2013
\end{etaremune}

		
\section*{Works in Progress}
\medskip
\begin{etaremune}
	\item How to be a Propositional Contingentist
	\item The Distribution Paradox
	\item Lost in Space
	\item Multi-Dimensional Quantified Modal Logic, under review
	\item The Paradox of Counterfactual Tolerance, under review
	\item Relational Possibility, under review
\end{etaremune}
	


		
\section*{Talks and Lectures}
\medskip
\begin{etaremune}
	\item ``Relational Possibility'', The Fourth Taiwan Metaphysics Colloquium, January 2020
	\item ``How to be a Propositional Contingentist'', guest lecture for Formal Methods in Metaphysics, a graduate seminar led by Hans Halvorson and Boris Kment, Princeton University, December 2019
	\item ``The Paradox of Counterfactual Tolerance'', University of Toronto philosophy department colloquium, January  2019 %25
	\item ``The Paradox of Counterfactual Tolerance'', The Fourth Taiwan Philosophical Logic Colloquium, November 2018
	\item ``Relational Possibility'', University of Victoria, Language and Essence conference, October 2018
	\item ``Relational Possibility'', Issues on the (Im)possible VI Conference, Institute of Philosophy, Slovak Academy of Sciences, August 2018
	\item Comments on Collier's ``God Exists in All Possible Worlds: Anselmian Theism and Genuine Modal Realism'', Issues on the (Im)possible VI Conference, August 2018
	\item Comments on Zee Perry's ``Mereology and Metricality'', 4th Annual Meeting of the Society for the Metaphysics of Science, Milan, Italy August 2018
	\item ``On Comparing'', Princeton University Philosophical Society, October 2013
	\item ``On Comparing'', Princeton University Dissertation Seminar, April 2013
	\item ``Molinism Without Manipulation'', Saint Thomas Summer Seminar in the Philosophy of Religion, July 2013
	\item Comments on Emanuel Viebahn's ``Invarientism about Might and May'', Oxford-Princeton Graduate Workshop, June 2013
	\item ``Beth Definability'', guest lecture for Hans Halvorson's Intermediate Logic, May 2013	
	\item ``A New Prospect for Creedal Aggregation'' (with Yoaav Isaacs), Rutgers-Princeton-Penn Social Epistemology Workshop, April 2013
	\item ``From Kinds to Oughts'', Princeton University Philosophical Society, April 2012
	\item ``Saving Ordinary Counterfactuals'', Oxford Graduate Conference, July 2013
	\item ``Saving Ordinary Counterfactuals'', Columbia-NYU Graduate Conference, April 2010
	\item ``Saving Ordinary Counterfactuals'', Harvard-MIT Graduate Conference, April 2010
	\item ``Saving Ordinary Counterfactuals'', Arché Graduate Conference, September 2009
	\item ``Disagreement About Epistemic Status'', Responsible Belief in the Face of Disagreement Conference, Vrije Universiteit, Netherlands, August 2009
\end{etaremune}

		
		
\section*{Teaching}
\medskip
\begin{etaremune}
	\item Introduction to Moral Philosophy (PHI 202), Princeton University, preceptor for Johan Frick, fall 2019
	\item Introduction to Logic (PHI 201), Princeton University, preceptor for Hans Halvorson, spring 2019
	\item Buddhist Philosophy (REL 281), Princeton University, preceptor for Jonathan Gold, spring 2017
	\item Introduction to Metaphysics and Epistemology (PHI 203), Princeton University, preceptor for Gideon Rosen, fall 2016 		
	\item Introduction to Metaphysics and Epistemology (PHI 203), Princeton University, preceptor for Gideon Rosen, spring 2016
	\item Introduction to Moral Philosophy (PHI 202), Princeton University, preceptor for Geoffery Sayer-McCord, fall 2015
	\item Philosophy of Mind (PHI 315), Princeton University, preceptor for Frank Jackson, \mbox{fall 2013}
	\item Intermediate Logic (PHI 312), Princeton University, preceptor for Hans Halvorson, spring 2012
	\item Theory of Knowledge (PHI 313), Princeton University, preceptor for Thomas Kelly, fall 2012
	\item Introduction to Logic, Brown University, assistant instructor for Richard Heck, \mbox{spring 2009}			
\end{etaremune}



\section*{Service to the Profession}
\medskip
\begin{etaremune}
\item[] Referee for Synthese, 2020-present
\item[] Referee for Epistemé, 2017-present
\item[] Referee for Philosophical Studies, 2012-present
\item[] Research Assistant for Boris Kment, 2013-14
\item[] Reviewer, Princeton-Rutgers Graduate Philosophy Conference, 2013
\item[] Reviewer, Brown Graduate Philosophy Conference, 2009
\item[] Reviewer, Brown Graduate Philosophy Conference, 2008
\item[] Research Assistant for Donald Wacome, 2005-06
\end{etaremune}        

%\section*{Dissertation Abstract: %\normalfont{Relational Possibility}}

\vspace{3pt}
%Turn off word splitting?

%\noindent Relational possibilities are about how things compare or otherwise relate across worlds. We might say that a pair of particles could have been farther apart than they are, for example. The usual view is that such possibilities require quantification over further things like numbers, spacetime points, or distances. The result, though, is that science would seem to be committed to more ontology---a wider range of things---than we might have otherwise thought. I develop a new view on which no such quantification is required. Modality, I say, is ultimately about how things could have \emph{differed}, not just how things could have been. Making sense of this idea means rethinking modal logic and modal metaphysics from the ground up. The result, though, is a conception of modality that not only better reflects our ordinary modal thought, but gives us powerful tools for doing science with nothing more than particles.
	
	\vfill \hfill CV Version: \today
	
\end{document}